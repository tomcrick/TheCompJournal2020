\documentclass[conference]{IEEEtran}
%\IEEEoverridecommandlockouts
% The preceding line is only needed to identify funding in the first
% footnote. If that is unneeded, please comment it out.
\usepackage[british]{babel}
\usepackage{cite}
\usepackage{amsmath,amssymb,amsfonts}
\usepackage{algorithmic}
\usepackage{graphicx}
\usepackage{tabularx}
\usepackage[pdftex,colorlinks=true]{hyperref}

\begin{document}

\title{Chains and Trees: Topical Patterns in Twitter Trends}

\author{\IEEEauthorblockN{Nabeel
    Albishry\IEEEauthorrefmark{1}\IEEEauthorrefmark{2}, Tom
    Crick\IEEEauthorrefmark{3} and Theo Tryfonas\IEEEauthorrefmark{1}}
\IEEEauthorblockA{\IEEEauthorrefmark{1}Faculty of Engineering, 
University of Bristol, Bristol, UK\\
Email: \{n.albishry,theo.tryfonas\}@bristol.ac.uk}
\IEEEauthorblockA{\IEEEauthorrefmark{2}Faculty of Computing \& IT, 
King Abdulaziz University, Jeddah, Saudi Arabia\\
Email: nalbishry@kau.edu.sa}
\IEEEauthorblockA{\IEEEauthorrefmark{3}Department of Computer
  Science, Swansea University, Swansea, UK\\
Email: thomas.crick@swansea.ac.uk}}

% \author{\IEEEauthorblockN{Nabeel Albishry}
% \IEEEauthorblockA{\textit{Faculty of Engineering} \\
% \textit{University of Bristol}\\
% Bristol, UK \\
% n.albishry@bristol.ac.uk}
% % \and
% % \IEEEauthorblockA{\textit{Faculty of Computing \& IT} \\
% % \textit{King Abdulaziz University}\\
% % Jeddah, Saudi Arabia \\
% % nalbishry@kau.edu.sa}
% \and
% \IEEEauthorblockN{Tom Crick}
% \IEEEauthorblockA{\textit{Department of Computer Science} \\
% \textit{Swansea University}\\
% Swansea, UK \\
% thomas.crick@swansea.ac.uk}
% \and
% \IEEEauthorblockN{Theo Tryfonas}
% \IEEEauthorblockA{\textit{Faculty of Engineering} \\
% \textit{University of Bristol}\\
% Bristol, UK \\
% theo.tryfonas@bristol.ac.uk}
% % \and
% % \IEEEauthorblockN{4\textsuperscript{th} Given Name Surname}
% % \IEEEauthorblockA{\textit{dept. name of organization (of Aff.)} \\
% % \textit{name of organization (of Aff.)}\\
% % City, Country \\
% % email address}
% }

\maketitle

\begin{abstract}
Thousands of topics trend on Twitter across the world every day,
making it challenging to provide in-depth analysis of current issues,
topics and themes being discussed across various locations and
jurisdictions. There is thus a demand for simple and extensible
approaches to provide deeper insight into these trends and how they
propagate across locales. Utilising graph structures, this paper
presents an exploration of topical patterns of trends on Twitter
across various regions. It is based on a year-long data collection
({\emph{N}}=2,307,163) and analysis between 2016-2017 of seven Middle
Eastern countries (Bahrain, Egypt, Kuwait, Lebanon, Qatar, Saudi
Arabia, and the United Arab Emirates). Using this year-long dataset,
the project identified two interesting structures of topics; chains
and trees. Trend topics that manifested themselves in these structures
found to represent ongoing concerns and interests.
\end{abstract}

\begin{IEEEkeywords}
Trends, topic structures, topic-trees, topic-chains, network graphs, Twitter
\end{IEEEkeywords}

% need to update Intro from ICCCI
\section{Introduction}\label{intro}

With the huge daily volume of generated content on Twitter -- c.500
million tweets per day -- trending topics serve as valuable sources of
information on highlighting what is going on in the world, or in
specific locations. Apart from the ``official'' trend lists provided
by the platform (on the website or through API endpoints), generating
insight from trends and topics detection has been receiving increasing
attention from across a variety of big social data-driven research
domains. In health for example, monitoring and analysis of trending
topics on social media has been adopted to measure emerging public
health issues, such as the spread of
influenza~\cite{Achrekar2011,Parker2013,Parker2015}. Furthermore,
across the marketing and business domains, topic detection and
classification are valuable approaches to extract knowledge and
insight on public opinions from posts on social
media~\cite{blamey-et-al-2012,Bello2013,albishry-et-al:ssei2018},
including analysing voting intentions and political view of
users~\cite{Fang2015}.

With the increasing popularity and use of social networks across a
wide range of domains, the impact of trends on public opinion and
perceptions has transformed social media campaigns and public
relations strategy. This has made trends a valuable target for
manipulation~\cite{Zhang2017}, stuffing~\cite{Irani2010},
spamming~\cite{Sedhai2015,Chu2012}, and
hijacking~\cite{VanDam2016}. Interestingly, deeper analysis of trend
hijacking cases suggests that increasing social media engagement may
not always be beneficial for public relations
strategies~\cite{Sanderson2016}.

A common approach in analysing Twitter trends is through clustering
and classification of trending topics based on
content~\cite{Zubiaga2011,Benhardus2013,Ferragina2015,albishry-et-al:iccci2017}.
The study in~\cite{TenThij2016} presented a content-independent method
to model trends progression through the dynamics of users
interactions; other studies have also attempted to provide real-time
classification or detection of
trends~\cite{Mathioudakis2010,Zubiaga2015}. With the increasing demand
for trends analysis across various domains, customisable clustering
tools that can be used by non-technical users have started to emerge,
such as the recent example introduced in~\cite{Arn2018}.

% need to update Methodology from ICCCI
\section{Methodology}\label{method}

The approach in this study consists of two main parts; graph construction
and text analysis. First, {\emph{weighted base graph}} is constructed to 
capture raw trends data in the form of country-trend relationships. Trend 
texts are then used to generate ngrams, and another {\emph{derived graph}}
is generated from these trend-ngram relationships. Different parts of these 
two graphs are then merged to link countries to topics of interests.

\subsection{Locales}

Seven Middle Eastern countries were selected for this study: Bahrain,
Egypt, Kuwait, Lebanon, Qatar, Saudi Arabia, and the United Arab
Emirates (UAE). The selection includes countries with relatively large
population (e.g. Egypt: 97,553,000) and relatively small populations
(e.g. Bahrain:
1,493,000)~\cite{UnitedNationsDepartmentofEconomicandSocialAffairs2017}.
Kuwait is reported to have the most active daily users on
Twitter~\cite{Salem2017}; as of March 2016, Saudi Arabia and Egypt
generated 33\% and 20\% of the tweets in the Middle Eastern
region. Bahrain is the most balanced location in terms of gender
breakdown of active users. Interestingly, between March 2014 and March
2016, Lebanon was the only location in the Middle Eastern states that
has not seen growth in active users, while UAE increased by 60\%. The
Gulf Cooperation Council countries -- Bahrain, Kuwait, Qatar, UAE, and
Saudi Arabia -- were reported to have the highest penetration
rates~\cite{Salem2017}.

\subsection{Data Collection}

Trending topic lists in the seven countries were monitored for a year
between October 2016 and October 2017. Every hour, trending lists were
collected through the Twitter REST API, which resulted in 7,948 hour's
worth of records for all the countries, totalling 2,307,163 trend
records. It is important to note that the Twitter API does not
necessarily provide trends data for every request; for example, it is
possible to receive no information for tweet volume. For each
location, the list of available trending topic is returned. From this
list, two pieces of information are extracted from each trend record;
{\emph{woeid}}: the Yahoo! Where On Earth ID (WOEID) of the location;
{\emph{name}}: text of trending topic (e.g
`{\texttt{\#Call\_For\_Action}}').


While the Twitter API returns a list of trending topics for a specific
{\emph{woeid}} location, the tweet volumes do not provide a
comprehensive measure of the tweeting activity in that
location. Rather, the tweet volume refers to the overall number of
tweets containing the trend, regardless of their location. Although
the Twitter
documentation\footnote{\url{https://developer.twitter.com/en/docs/trends/trends-for-location/api-reference/get-trends-place}}
does not provide the necessary detail on this, it was apparent after
observing trends that showed up in various locations. Trends were
found with the same tweet volume across all locations and, hence,
participation volume of each location was not possible to be
accurately measured. Therefore, the context of this study does not
include any reference to this volume entity.

\subsection{Base Graph Construction: Temporal edges}

This directed graph was constructed from raw data. The main use of this graph was to measure the temporal appearance order of the trends for places. As the graph was aimed to mainly capture data structures, constructing this graph was a simple process.
In this graph each edge had a timestamp of the trend appearance, therefore the connected pair of nodes might possess more than one edge, as shown in Figure~\ref{fig:basegraph}. The participation of the country measures the number of trends it generates, hence, two types of nodes were included in this graph: country and trend. It is important to note that the network was constructed whereby the edges originated from the country node toward the trend node. Although an undirected graph could be used, the direction orientation was important for distinguishing the node types by their indegrees. The nodes with a zero indegree identified countries, while the indegree of the trends nodes could not be less than one. Changing the direction of the edge would provide the same results, and in this case, inverted centrality measures were used. For example, the indegree was used instead of the outdegree, and vice versa. 

\begin{figure}[htb] \centering
\includegraphics[width=\columnwidth]{images/base_graph.png}
\caption{Illustration of the base graph}
\label{fig:basegraph}
\end{figure}


\subsection{Derived Graph Construction: Weighted edges}

This graph is generated from the base graph, following the algorithm 
shown in Table~\ref{tbl:algorithm1}. It is a directed graph and that consists of two trend entities:
{\emph{place}} and {\emph{trend}}. Nodes represent places and
trends, and are linked with {\emph{weighted}} edges. The feature 
of weighted edges in this graph is used to measure the popularity 
of trends, repetition rate, participation of countries, and the volume 
of the engagement. Figure~\ref{fig:w_graph} shows an example 
of such graph.


\begin{table}
\centering
\begin{tabular}{l|l}
\hline
{\emph{1}} & weighted\_graph = empty\_graph()\\
{\emph{2}} & \textbf{For} edge \textbf{in} base\_graph.edges():\\
{\emph{3}} &  $\>\>\>\>$\textbf{if} weighted\_graph.has\_edge(country, trend):\\
{\emph{4}} &  $\>\>\>\>\>\>\>\>$ edge\_weight+=1\\
{\emph{5}} &  $\>\>\>\>$\textbf{else:}\\
{\emph{6}} &  $\>\>\>\>\>\>\>\>$ add\_edge(country, trend)\\
\hline
\end{tabular}
\caption{Trend-Ngram graph construction algorithm}
\label{tbl:algorithm1}
\end{table}


%\begin{figure}[htb] \centering
%\includegraphics[width=\columnwidth]{images/algorithm1.png}
%\caption{Country-Trend weighted graph construction algorithm}
%\label{fig:algorithm1}
%\end{figure}


\begin{figure}[htb] \centering
\includegraphics[width=\columnwidth]{images/w_graph.png}
\caption{Example of weighted graph}
\label{fig:w_graph}
\end{figure}


\subsection{Text analysis and Ngraming}

Trends exist in the form of phrases or hashtags. The main textual
difference between these is that hashtags appear as a connected chain
of characters with no whitespaces, while trending phrases include
spaces. To create a hashtag with more than one word, it must be
possible to distinguish the words while maintaining their
connectivity, and Twitter allows underscores to be used in multiple
word hashtags. For languages such as English, letter capitalisation is
an additional popular option in social media for hashtagging, such as
\#ThisIsAnExample. However, languages such as Arabic, Urdu, and Farsi,
do not employ capitalisation in the same way, therefore the only means
of separating words is to use underscores.  

The textual analysis in
this study required the trend texts to be tokenised to generate
ngrams. This process was straightforward for the phrase trends,
although the hashtag trends included a number of variations that
required consideration during the textual processing. In capitalised
English hashtags, for example, it was unclear whether the capital
letters referred to the beginning of words or abbreviations. In
contrast, Arabic hashtags use underscores to separate words, which
therefore make word separation an easy and accurate task to
accomplish. As the sample countries were from the Arab region, it was
anticipated that most tweeting activity would occur in the Arabic
language, since an earlier study demonstrated that 72\% of all tweets
in the Arab region are in Arabic [94].

In this present study, the textual analysis accounted for underscores
and spaces only where they occurred as word separators. While this
implied that capitalised hashtags should be considered as a single
word, this would not provide an accurate analysis for this group of
trends, but since Arabic trends were dominant in the dataset, and only
capitalised hashtags were affected, this was not anticipated to have
an impact on the results of the analysis.


\subsection{Derived Graph: Trend-Ngrams}

This directed graph utilised the trend nodes in the base graph to
generate its nodes and edges. It included two types of nodes: trend
and ngram. To generate ngrams from a trend, the trend first required
tokenisation. The trends displayed in two forms, either as a phrase,
such as 'coffee international day', or a hashtag such as
'{\texttt{\#coffee\_international\_day}}'.  Phrase trends can be
tokenised from their original form, while hashtags do not include
spaces, and hence appear as one word, and therefore first required
processing in order that all of the trends were translated into phrase
form before being tokenised. The hashtags processing involved
stripping them of the '\#' sign, and replacing the underscores with
spaces. Then, preserving the word order, the phrases were tokenised,
and all possible ngrams then generated.  The original trend was
connected to all of the ngrams generated, and an additional length
attribute was added to each node within the graph, as shown in lines
five and seven in Table~\ref{tbl:algorithm2}. 
This attribute was used later in the
analysis and the visualisation setting. Applying this algorithm to the
hashtag '{\texttt{\#coffee\_international\_day}}' resulted in the
graph shown in Figure~\ref{fig:tngraph}.

\begin{table}
\centering
\begin{tabular}{l|l}
\hline
{\emph{1}} & trend\_ngram\_graph = empty\_graph()\\
{\emph{2}} & \textbf{For} node \textbf(in) base\_graph.nodes():\\
{\emph{3}} &  $\>\>\>\>$ \textbf{if} indegree(node) = 0:\\
{\emph{4}} &  $\>\>\>\>\>\>\>\>$tokens = tokenize(node)\\
{\emph{5}} &  $\>\>\>\>\>\>\>\>$trend\_ngram\_graph.add\_node(node, length(tokens))\\
{\emph{6}} &  $\>\>\>\>\>\>\>\>$\textbf{For} n \textbf{in} all\_ngrams(tokens):\\
{\emph{7}} &  $\>\>\>\>\>\>\>\>\>\>\>\>$ trend\_ngram\_graph.add\_node(n, length(n)):\\
{\emph{8}} &  $\>\>\>\>\>\>\>\>\>\>\>\>$ \textbf{if} trend\_ngram\_graph.has\_edge(node, n):\\
{\emph{9}} &  $\>\>\>\>\>\>\>\>\>\>\>\>\>\>\>\>$ edge\_weight += 1\\
{\emph{10}} & $\>\>\>\>\>\>\>\>\>\>\>\>$ \textbf{else:}\\
{\emph{11}} & $\>\>\>\>\>\>\>\>\>\>\>\>\>\>\>\>$ add\_edge(node, n)\\
\hline
\end{tabular}
\caption{Trend-Ngram graph construction algorithm}
\label{tbl:algorithm2}
\end{table}

% \begin{figure}[htb] \centering
% \includegraphics[width=\columnwidth]{images/algorithm2.png}
% \caption{Trend-Ngram graph construction algorithm}
% \label{fig:algorithm2}
% \end{figure}

\begin{figure}[htb] \centering
\includegraphics[width=\columnwidth]{images/trend_ngram_example.png}
\caption{Trend-Ngram graph example}
\label{fig:tngraph}
\end{figure}

A trend node in the base graph cannot be repeated. Therefore, the
edges linking the trend to the ngrams in this graph were expected to
have a fixed weight of 1.  However, observation of the edge weights
revealed edges with higher weights, namely 2 and 3, occurring in
trends with repeated words, such as
`{\texttt{\#welcome\_you\_welcome}}'.  In total, 264 trends had two
repeated words, while five trends had three. Since this small number
of cases fell outside the context of this study, and the analysis of
this graph did not utilize individual edge weights, they did not
affect the results. As a result, only the indegree and outdegree
centrality measures were employed for this graph, with both the
indegree of all of the trend nodes, and the outdegree of the ngram
nodes being 0. These two measures were employed in the analysis phase
to distinguish the node types. Furthermore, the indegree of the ngram
nodes reflected the number of the connected trend, and was at least
1. The outdegree of the trend nodes indicated the number of related
ngrams. As the ngrams were generated from the trends, the outdegree of
the trend was found to be fixed, and was equivalent to $(n(n+1))/2$,
where n is length of trend.

\section{Results}\label{results}

Observation of the weighted graph provided an overall evaluation of
activity for trends and places. In total there were 76,266 distinct
trends that trended 2,307,163 times across all locations; this
suggests that trends may appear repeatedly over time. The overall
repetition ratio in the dataset was 97\%, and ranged from 80\% to 98\%
for individual locations, with Saudi Arabia scoring lowest and Qatar
scoring highest rate. {\emph{Indegree}}, {\emph{outdegree}}, and
{\emph{edges}} were used to conduct subsequent results, with further
explanation to follow in the relevant sections.

\subsection{Textual features of trends}

As can be seen in Figure~\ref{fig:trends_lengths_wind}, two-word
trends were the most common trends. Presence of 1 and 3-word trends
were at similar level, and 4-word trend are a little lower. Then,
lengths started to show dramatic drop from 5-word
onwards. Additionally, weighted indegree of trend was found strongly
correlated with the word length.

\begin{figure}[htb] \centering
\includegraphics[width=\columnwidth]{images/trends_lengths_wind.png}
\caption{Ratios of trend lengths and their weighted indegree}
\label{fig:trends_lengths_wind}
\end{figure}

A single odd case was found in which processing the trend’s text
resulted in 16 words, which indicate very unusual case. Further
investigation uncovered that the trend consists of three words that
were intentionally letter spaced, seemingly for fun. To clarify it
further, the trend was in Arabic and it is the equivalent of
{\texttt{\#WriteInSeparateLetters} but was written as
{\footnotesize{{\texttt{\#w\_r\_i\_t\_e\_I\_n\_S\_e\_p\_a\_r\_a\_t\_e\_L\_e\_t\_t\_e\_r\_s}}}}.
Therefore, this was removed from the dataset before progressing to
further analysis. This demonstrates additional benefit of analysing
length of trend in removing such odd cases.

\subsection{Chain and Tree topics}

Investigation of trend-ngram graph uncovered two special group of trends 
that appeared in form of trees and chains. 
Each form features trends that relate to same ngram but longer by exactly 
one segment in terms of word numbers. This last segment is found to be 
the key to distinguish trend trees from chains. Last segments of all related 
trends are examined to confirm if they are digit or words. Then, chain 
likelihood is calculated by measuring the ratio of digits to words is measured. 
For example, “\#Call\_for\_action\_1”, “\#Call\_For\_Action\_2”, and 
“\#Call\_for\_action\_3” 
form a trend chain and relate to the ngram “Call,For,Action”. An example 
of a trends tree graph would be the trends “russain\_foriegn\_minister” and 
“american\_foriegn\_minister” relating to ngram “foriegn,minister”. Indegree of 
the ngram indicates number of relating trends and measure length of the 
trend chain or number of topic branches. Having a perfect chain with high indegree 
is very likely to uncover well-planned ‘Twitter campaign’, regardless 
of its aims or representativeness of public concerns on the real world.
The example in Figure~\ref{stc_chain} represents a perfect chain with bigram 
origin node. The example is related to a well-known boycott campaign that 
started early October 2016 against the Saudi Telecommunication Company (STC) ~\cite{naffee-2016}.
On the other hand, when value of chain 
likelihood is found 0, some perfect trend trees were identified. 
Representation of a perfect trends tree is shown in Figure~\ref{fig:ministers_tree}. 
The core of the graph is the bigram ‘forign,minister’, and is connected to 25 
trends, such as ‘saudi\_foreign\_minister’ and ‘iranian\_foreign\_minister’. As can be seen, 
trend trees were found featruing more general topics compared to trend 
chains that feature more specific topics or issues.

\begin{figure}[htb] \centering
\includegraphics[width=\columnwidth]{images/foreign_ministers_tree.png}
\caption{Topic tree for foreign ministers trends}
\label{fig:ministers_tree}
\end{figure}

\begin{figure}[htb] \centering
\includegraphics[width=\columnwidth]{images/stc_chain.png}
\caption{Topic chain for STC boycott trends}
\label{fig:stc_chain}
\end{figure}

\subsection{Merging graphs: Country-Trend-Ngram}

Chain and tree graphs can be extended to provide more insights on participation 
of countries. Since the example in Figure~\ref{fig:ministers_tree} is relatively low in size, its visual 
representation will be clearer to present here. The ngram-trends graph above 
was merged with related edges from the base graph (country-trend). 
This has resulted in more detailed graph, by which visualisation became more 
informative, as can be seen in Figure~\ref{fig:ctn_graph} below. This graph inherits nodes and 
edges from the combined graphs (country-trend and trend-ngram). Therefore, 
it includes three types of nodes; country, trend, and ngram, and two kinds of 
directed edges; country-trend and trend-ngram. So, paths flow as country-trend-ngram. 

Because there is one ngram node only, one colour and fixed size are used to distinguish it (green). For country nodes, shades of red are used for weighted outdegree, while node size reflects outdegree. Whereas for trend nodes, purples shades are used for weighted indegree and size indicates indegree.

\begin{figure}[htb] \centering
\includegraphics[width=\columnwidth]{images/ctn_graph.png}
\caption{Foreign ministers tree merged with weighed country-trend graph}
\label{fig:ctn_graph}
\end{figure}

\section{Discussions}\label{discussion}

The analysis of trends showed a high repetitiveness in the more general trends, 
such as ‘\#good\_morning’, ‘\#Friday’, and ‘\#Trump’. Nevertheless, the topic 
ngrams helped to group the trends to provide a broader view. The trend-ngram 
graph facilitated the measurement of various forms of related trends. 
For example, the bigram (‘We will, bankrupt you,’) was found in 
relation to 232 trends. As ngrams are inclusive, the general graphs favoured 
smaller ngrams, such as unigrams and bigrams, over larger forms. Therefore, 
the ngram-specific graphs were more useful in conducting a more accurate 
analysis, and could be tailored, based on the breadth of analysis required. 
For example, the trend-unigram graph helped to extract all of the trends relating 
to a specific word, while the trend-trigram graph provided a more specific view of a topic.
   
Meanwhile, using the chain likelihood measure in the trend-ngram graph helped 
to identify certain important topics and their graph structures. Although many 
related trends had chain likelihood values between 0 and 1, some important topics 
emerged in the perfect form of chains or trees, 0 for trees, or 1 for chains. 
The longest chain was found to be 
(Saudi\_women\_request\_guardianship\_omission), which had links to 224 trend 
versions. Trend chains were useful for identifying frequently trending topics, 
and reflected a good degree of organization in promoting and running social 
media campaigns. Furthermore, the trend trees graph structures helped to identify 
the subtopics of related trends, and their root topic. In addition to the tree example 
in Figure~\ref{fig:ministers_tree}, the bigram (‘international\_day’) 
was found to be linked to 64 trends representing different celebrated days on Twitter. Other perfect forms 
of trees included matches for football clubs, such as (‘Real\_madrid’), and 
celebrity birthdays (‘birthday’). 
The overall view of the chains and trees was that the trend chains possessed a 
consistent textual structure that constituted a form of tweeting consensus over time, 
and therefore increased the likelihood of further similar trends appearing in the chain. 
Nevertheless, the continuity and length of the chain depended on other factors, such 
as real-world developments. In contrast, the trend trees were generic, and did not 
include a textual pattern, apart from containing the root ngram. 
Therefore, for a social media campaigns that included a continuous trend generation,
or trend chain, it was important to maintain a consistent chain textual pattern. Accordingly,
branching a chain in terms of turning chains into trees, disturbed their consistency, and 
would engender a dispersed tweeting effort, hence lowering the chance of long chains. 
This highlighted a vulnerability that might be exploited as a countermeasure, the existence 
of which might be observed in anti-government and boycott campaigns. Although this 
claim was not testified in this present study, a suggestion for future analysis would be to 
measure the temporal changes of the graph structures, and the chain likelihood value.  


\section{Conclusions}\label{conclusions}
This paper presented a mixed approach for trend analysis using graph construction and ngram textual analysis. The techniques were tested on one-year’s worth of trend data for seven Middle Eastern Arabic-speaking countries. The analysis of the data employed one temporal base graphs, and two derived graphs. 
First, a directed base graph was generated to capture the countrytrend relationships. For the trend nodes in this graph, other information, such as timestamp, was added as node attributes. This graph was used to measure trend popularity and the participation of the sample countries, together with exploring the appearance order of trends across countries, and measuring the countries’ tendency to engage with common trends. From this graph, an undirected country-country graph was derived to explore the relationship between the countries, in the form of common trends.
Second, ngrams were generated from the trends to construct another directed base graph. The nodes in this graph represented trends and ngrams, and edges were created based on the trendngram relationships. The main attribute of these nodes was their word length, which was used to identify the chains and trees in the trends. Although the results showed perfect chain and tree structures, other mixed forms of structures could also be identified. For example, a tree of chains might be identified to provide another perspective of the results.
Third, the two base graphs were combined to generate another directed graph, capturing both the countrytrend, and trendngram relationships. Although the example shown was a perfect tree, this graph could be used to observe the contribution of countries in chains, or any other topic of interest. 
Finally, the visual representation of the graphs was produced in three forms. First, a whole graph figure was important for generating graphs such as country-country. Although this graph could be filtered to explore the relationships between certain countries, filtering such a graph would not correctly represent the overall centrality measures, such as the node degrees and edge weights. Second, due to their size, graphs such as countrytrend, and trendngram required filtering, which was applied to include the important nodes, their edges, and related nodes, based on their centrality measures. Some examples visualized included the top 10\% of nodes with a high-weighted indegree. 
The last form of visualization was based on the node of interest. The example in Figure~\ref{fig:ministers_tree} 
were produced for hand-picked nodes, while the core node of the graph shown in Figure~\ref{fig:ctn_graph} was selected, and the related edges and nodes were then extracted before visualizing the whole graph. 
In conclusion, the approach presented can employed to measure broad public concerns, and prolonged matters, as well as the possible spread of trends, based on their historical records as well as their originating country. The flexibility of this approach means it is appropriate for the aim of analysis. 
A real-time application of this approach would be useful for observing certain trend activities, such as studying trend data in order to capture trends relating to a predefined ngram for a particular group of countries, or to discover emerging trend chains or trees.

%\section*{Acknowledgment}

% BibTeX users
\bibliographystyle{IEEEtran}      % basic style, author-year citations
\bibliography{asonam2018}

\end{document}
